%
% do not add anything in this part
%
\documentclass[catalog.tex]{subfiles}

% bib file: write the bibliography in a file having the same name as the latex file
% change the suffix".tex" by ".bib"
% biblatex is required, do not use bibtex!
% DO NOT TOUCH THE FOLLOWING LINE

\begin{document}

%
% things can be added below
%

\def\pbname{PNG (Encoding and Decoding)} %change this, do not use any number, just the name

\section{\pbname}

% only for overview, so short description (no more than 1-2 lines)
\begin{overview}
\item [Algorithm:] PNG (Encoding and Decoding)~(algo.~\ref{alg:\currfilebase})
	% -	must match the label of the algorithm
	% - when writing more than one algo use alg:\currfilebase_a, alg:\currfilebase_b, etc.
\item [Input:] PNG image
\item [Complexity:] $\mathcal{O}(n\log{n})$
\item [Data structure compatibility:] Huffman tree
\item [Common applications:] Face recognization
\end{overview}


\begin{problem}{\pbname}
	PNG\cite{Ch.1}, ``Portable Network Graphics'', is a computer file format for storing, transmitting, and displaying images. It has features of lossless compression, transparency information, interlacing and a range of color depths. We analyze PNG's file structure, lossless filtering and compression to encode and decode PNG.
\end{problem}


\subsection*{Description}
PNG format is widely used now. Comparing with other image format like JPG, BMP, the advantage of PNG is smaller file size and that's why it's called ``portable''. And the advantage benefits from its lossless filtering and compression. We will introduce the file structure of PNG, lossless filtering and compression algorithm.

	\subsubsection*{File Structure}
	The fundamental building block of PNG images is the {\it chunk}\cite{Ch.8}. A PNG file consists of PNG signture and chunks. PNG format defines two kinds of chunk, {\it critical chunk} and {\it ancillary chunks}. It is essential for PNG encoding and decoding softerwares to support critical chunks encoding and decoding. Hence we will focus on critical chunks. A general PNG file consists of PNG signture and three critical chunks, IHDR (header chunk), IDAT (image data chunk), IEND (image trailer chunk) (Tbl.~\ref{tbl:\currfilebase_1}). The components are in this order.


	\begin{table}[!htb]
		\caption{General PNG file structure}
		\label{tbl:\currfilebase_1}
		\centering
		\begin{tabular}{cccc}
			\toprule
			\multicolumn{4}{c}{PNG File} \\
			\midrule
			PNG Signature & IHDR & IDAT & IEND \\
			\bottomrule
		\end{tabular}
	\end{table}


	\begin{enumerate}
		\item {\bf PNG Signature}\\
		PNG Signture is the first 8-byte of a PNG file, which is just a simple identifier code. It is designed to detect the most common types of file-transfer corruption. The contents of PNG signature are {\tt 0x89, 0x50, 0x4E, 0x47, 0x0D, 0x0A, 0x1A}(Tbl.~\ref{tbl:\currfilebase_2}).


		\begin{table}[!htb]
			\caption{PNG Signature Bytes\cite{Ch.8}}
			\label{tbl:\currfilebase_2}
			\centering
			\begin{tabular}{cl}
				\toprule
				Decimal Value & ASCII Interpretation\\
				\midrule
				137 & A byte with its most significant bit set (``8-bit character'') \\
				\midrule
				80 & P \\
				\midrule
				78 & N \\
				\midrule
				71 & G \\
				\midrule
				13 & Carriage-return (CR) character, a.k.a. CTRL-M or \^\ M \\
				\midrule
				10 & Line-feed (LF) character, a.k.a. CTRL-J or \^\ J \\
				\midrule
				26 & CTRL-Z or \^\ Z \\
				\midrule
				10 & Line-feed (LF) character, a.k.a. CTRL-J or \^\ J \\
				\bottomrule
			\end{tabular}
		\end{table}


		\newpage
		\item {\bf Chunks}\\
		Although there are chunks identified as different types, they are bulit in common structure. Each chunk can be divided into 4 parts (in sequence) (Tbl.~\ref{tbl:\currfilebase_3}).


		\begin{table}[!htb]
			\caption{Chunk structure}
			\label{tbl:\currfilebase_3}
			\centering
			\begin{tabular}{ccc}
				\toprule
				Name & Bytes & Description \\
				\midrule
			  Length & 4 & Length of Chunk data, not longer than ($2^{31}-1$)bytes \\
				\midrule
				Chunk Type Code & 4 & Consists of ASCII character \\
				\midrule
				Chunk data & Determined by Length & Main data of chunk \\
				\midrule
				CRC & 4 & Cyclic redundancy check\\
				\bottomrule
			\end{tabular}
		\end{table}


		\begin{itemize}
			\item IHDR (header chunk)\\
			IHDR stores the basic information of image data. It has to be the first chunk and the only one. The chunk data of IHDR is 13-byte long. It determines the width and height of the image, bit depth, color type, compression method, filter method, interlace method.
			\item IDAT (image data chunk)\\
			IDAT stores the main image data of PNG file. Then compression and filtering method are applied to generate this chunk.
			\item IEND (image trailer chunk)
			IEND marks the end of PNG file. It has to be the end of the file. The contents of it are {\tt 0x00, 0x00, 0x00, 0x00, 0x49, 0x45, 0x4E, 0x44, 0xAE, 0x42, 0x60, 0x82}, which corresponds to the chunk structure.
		\end{itemize}
	\end{enumerate}


	\subsubsection*{Filtering method\cite{Ch.9}}
	The filtering method of PNG is differential coding, which is reversible. The data of a pixel is determined by the pixels on the upper, left, upper-left of the pixel. PNG provides 5 ways to calculate the substraction, which is determined by the filter method byte (Tbl.~\ref{tbl:\currfilebase_4}).


	\begin{table}[!htb]
		\caption{Filtering method}
		\label{tbl:\currfilebase_4}
		\centering
		\begin{tabular}{cc}
			\toprule
			Name & Description \\
			\midrule
			None & Unchanged \\
			\midrule
			Sub & Difference between the left pixel and itself \\
			\midrule
			Up & Difference between the upper pixel and itself \\
			\midrule
			Average & Difference between the averge of upper and left pixels and itself \\
			\midrule
			Paeth & Difference between the left, upper, upper-left pixels and itself\\
			\bottomrule
		\end{tabular}
	\end{table}

	\newpage
	\subsubsection*{Compression method}
	The compression method of PNG is deflate algorithm, which is a mixture of Huffman coding (Alg.~\ref{alg:\currfilebase}) and LZ77.
	\begin{Algorithm}[Huffman coding \label{alg:\currfilebase}]
		% -	must match the reference in the overview
		% - when writing more than one algo use alg:\currfilebase_a, alg:\currfilebase_b, etc.
		%\SetKwFunction{myfunction}{MyFunction}
		\Input{Symbols}
		\Output{Encoded symbols}
		%	\Fn{\myfunction{$a,b$}}{
		%	}
		\BlankLine
		Calculate the probablities of each different symbols\\
		Assign the probability to each node\\
		Sort the probabilities in non-decreasing order\\
		\While{all probablities of nodes in the tree are less than 1}
		{
			Generate a parent for the least and the second least probablity nodes\\
			Assign the sum of the probablities of these two childs to the parent\\
			Mark the child with less probability as 0\\
			Mark the child with less probability as 1\\
		}
		The path to the symbol (consisting of zeros and ones) is Huffman code of the symbol.\\
		\Ret Pathes

	\end{Algorithm}


% include references where to find information on the given problem using latex bibliography
% insert references in the text (\cite{}) and write bibliography file in problem-nb.bib (replace nb with the problem number)
% prefer books, research articles, or internet sources that are likely to remain available over time
% as much as possible offer several options, including at least one which provide a detailed study of the problem
% if available include links to programs/code solving the problem
% wikipedia is NOT acceptable as a unique reference
\singlespacing
\printbibliography[title={References.},resetnumbers=true,heading=subbibliography]

\end{document}
