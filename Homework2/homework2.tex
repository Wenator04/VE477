\documentclass[a4paper]{article}
\usepackage{ulem}
\usepackage{graphicx}
\usepackage[namelimits]{amsmath}
\usepackage{amssymb}
\usepackage{amsmath}
\usepackage{amsfonts}
\usepackage{mathrsfs}
\usepackage{enumerate}
\usepackage{indentfirst}
\usepackage{multirow}
\usepackage{latexsym}
\usepackage{subfig}
\usepackage{listings}
\usepackage{xcolor}
\usepackage{algorithm}
\usepackage{algpseudocode}
\usepackage{forest}

\lstset{numbers=left,
	numberstyle=\tiny,
	frame=shadowbox,
	backgroundcolor=\color[RGB]{245,245,244},
	keywordstyle=\color[RGB]{40,40,255},
	numberstyle=\footnotesize\color{darkgray},
	commentstyle=\color[RGB]{50,50,50},
	breaklines=true}

\renewcommand{\algorithmicrequire}{\textbf{Input:}}
\renewcommand{\algorithmicensure}{\textbf{Output:}}

\title{UM-SJTU JOINT INSTITUTE\\VE477 Introduction to Algorithms\\\vspace{0.5cm} Homework 2}
\author{Li Yong 517370910222}
\begin{document}
\maketitle
\newpage

\section*{Ex.1 Basic complexity}
	\begin{enumerate}[1.]
		\item
		\begin{enumerate}[a)]
			\item If $$n^3-3n^2-n+1=\Theta(n^3)$$
			Then there exists $c_1$, $c_2$, $n_0$ such that $\forall n\geq n_0$
			$$c_1n^3\leq n^3-3n^2-n+1 \leq c_2n^2$$
			$$c_1\leq \frac{1}{n^3}-\frac{1}{n^2}-\frac{3}{n}+1 \leq c_2$$
			When $n_0 = 4$, $c_1 = 0.05$, $c_2 = 1$, it satisfies.
			\item If $$n^2 = O(2^n)$$
			Then there exists $c_1$, $n_0$ such that $\forall n\geq n_0$
			$$n^2 \leq c_1 2^n$$
			When $n_0 = 4$, $c_1 = 1$, it satisfies.
			\item If $$\forall a\in \mathbb{R}, b\in \mathbb{R^+}, (n+a)^b = \Theta(n^b)$$
			Then there exists $c_1$, $c_2$, $n_0$ such that $\forall n\geq n_0$
			$$c_1 n^b\leq (n+a)^b \leq c_2 n^b$$
			$$c_1\leq (\frac{n+a}{n})^b \leq c_2$$
			When $a \geq 0$, $n_0=2a$, $c_1 = 1$, $c_2=1.5^b$, it satisfies.\\
			When $a < 0$, $n_0=-2a$, $c_1 = 0.5^b$, $c_2=1$, it satisfies.
		\end{enumerate}
		\item
		\begin{enumerate}[a)]
			\item $f(n) = O(g(n))$
			\item $f(n) = \Omega(g(n))$
		\end{enumerate}
		\item
		\begin{enumerate}
			\item None.
			\item $f(n)=2^n-n^2$, $g(n)=n^4+n^2$
		\end{enumerate}
		\item
		Considering $f_2$ and $f_3$, which are easy to compare
		$$\frac{f_3(n)}{f_2(n)} = \sqrt{\frac{n}{\log{n}}}$$
		Apparently, $f_2(n)<f_3(n)$.
		$$f_1(n) = \sum_{i=1}^n\sqrt{i} > \frac{n(1+\sqrt{n})}{2}$$
		$$(1+\sqrt{n})^2 - (2\sqrt{\log{n}})^2 = n + 2\sqrt{n} + 1 - 4\log{n} > n + 1 - 2\log{n}>0$$
		Hence, $$f_1(n) > \frac{n(1+\sqrt{n})}{2} > n\sqrt{\log{n}} = f_3(n)$$
		$$f_4(n) = \Theta(n\sqrt{n})$$
		Hence, $$f_4>f_1>f_3>f_2$$
	\end{enumerate}

\section*{Ex.2 Master Theorem}
	\begin{enumerate}[1.]
		\item
		\begin{enumerate}[a)]
			\item
			\begin{forest}
				[$f(n)$
					[$f(n/b)$
						[$f(n/b^2)$]
						[$b$ childs in total]
						[$f(n/b^2)$]
					]
					[$\cdots$]
					[$b$ childs in total]
					[$\cdots$]
					[$f(n/b)$
						[$f(n/b^2)$]
						[$b$ childs in total]
						[$f(n/b^2)$]
					]
				]
			\end{forest}
			$$\cdots$$
			\item
			\begin{enumerate}[i - ]
				\item $\log_bn+1$
				\item $a^{\log_b n}$
				\item $a^kf(n/b^k)$
				\item
				\begin{align*}
					T(n)&=\sum_{j=0}^{log_bn}a^jf(\frac{n}{b^j})\\
					&=a^{\log_bn}f(1) + \sum_{j=0}^{log_bn-1}a^jf(\frac{n}{b^j})\\
					&=n^{\log_ba}f(1) + \sum_{j=0}^{log_bn-1}a^jf(\frac{n}{b^j})\\
					&=\Theta(n^{\log_ba}) + \sum_{j=0}^{log_bn-1}a^jf(\frac{n}{b^j})
				\end{align*}
			\end{enumerate}
		\end{enumerate}
		\item
		\begin{enumerate}[a)]
			\item
			\begin{enumerate}[i - ]
				\item If $$g(n) = \Theta\left(\sum_{j=0}^{log_bn-1}a^j(\frac{n}{b^j})^{\log_ba}\right)$$
				Then there exists $c_1$, $c_2$, $n_0$ such that $\forall n\geq n_0$
				$$c_1\sum_{j=0}^{log_bn-1}a^j(\frac{n}{b^j})^{\log_ba}\leq g(n) \leq c_2\sum_{j=0}^{log_bn-1}a^j(\frac{n}{b^j})^{\log_ba}$$
				According to the previous question, $f(n)=\Theta(n^{\log_ba})$. Then there exists $n_0'$, $c_1'$, $c_2'$ such that $\forall n\geq n_0'$
				$$c_1'n^{\log_ba} \leq f(n) \leq c_2'n^{\log_ba}$$
				$$c_1'\left(\frac{n}{b^j}\right)^{\log_ba} \leq a^jf\left(\frac{n}{b^j}\right) \leq c_2'\left(\frac{n}{b^j}\right)^{\log_ba}$$
				Hence $\forall n_0' = n_0/b^j$, there exist $c_1$, $c_2$ such that
				$$c_1\sum_{j=0}^{log_bn-1}a^j(\frac{n}{b^j})^{\log_ba}\leq \sum_{j=0}^{log_bn-1}a^jf(\frac{n}{b^j}) \leq c_2\sum_{j=0}^{log_bn-1}a^j(\frac{n}{b^j})^{\log_ba},$$
				which is exactly the formula we obtained at first.
				\item
				\begin{align*}
					\sum_{j=0}^{log_bn-1}a^j(\frac{n}{b^j})^{\log_ba}&=n^{\log_ba}\sum_{j=0}^{log_bn-1}\frac{a^j}{(b^{\log_ba})^j}\\
					&=n^{\log_ba}\log_bn
				\end{align*}
				\item $$g(n) = \Theta\left(\sum_{j=0}^{log_bn-1}a^j(\frac{n}{b^j})^{\log_ba}\right)=\Theta(n^{\log_ba}\log_bn)$$
			\end{enumerate}
			\item
			\begin{enumerate}[i - ]
				\item Skipped, similar with 2-ii
				\item
				\begin{align*}
					\sum_{j=0}^{log_bn-1}a^j(\frac{n}{b^j})^{\log_ba-\epsilon}&=n^{\log_ba-\epsilon}\sum_{j=0}^{log_bn-1}\frac{a^j}{(b^{\log_ba-\epsilon})^j}\\
					&=n^{\log_ba-\epsilon}\sum_{j=0}^{log_bn-1}b^{\epsilon j}\\
					&=n^{\log_ba-\epsilon}\frac{(b^{\log_bn})^\epsilon-1}{b^\epsilon-1}\\
					&=\frac{n^\epsilon-1}{b^\epsilon-1}n^{\log_ba-\epsilon}
				\end{align*}
				\item $$g(n)=O\left(\sum_{j=0}^{\log_bn-1}a^j(\frac{n}{b^j})^{\log_ba-\epsilon}\right)=O\left(\frac{n^\epsilon-1}{b^\epsilon-1}n^{\log_ba-\epsilon}\right)$$
			\end{enumerate}
			\item
			\begin{enumerate}[i - ]
				\item $$\sum_{j=0}^{log_bn}a^jf(\frac{n}{b^j}) = f(n) + \sum_{j=1}^{log_bn}a^jf(\frac{n}{b^j}) > f(n)$$
				Hence, $g(n) = \Omega(f(n))$.
				\item Consider $j=1$, then $af(a/b)\leq cf(n)$ is true.\\
				We assume that when $j=k$, the formula is true. Then as for $j=k+1$
				$$a^{j+1}f(n/b^{j+1}) = a^{j+1}f((n/b^j)/j)\leq a^{j}cf(n/b^j)\leq c^{j+1}f(n)$$
				\item $$a^jf(n/b^j)\leq c^jf(n)$$
				$$\sum_{j=0}^{\log_bn-1}a^jf(n/b^j)\leq \sum_{j=0}^{\log_bn-1}c^jf(n)$$
				$$g(n)\leq\sum_{j=0}^{\log_bn-1}c^jf(n)$$
				$$g(n)=O(f(n))$$
				\item $g(n)=\Omega(f(n)), g(n)=O(f(n)) \Rightarrow g(n)=\Theta(f(n))$
			\end{enumerate}
		\end{enumerate}
		\item As for the Master Theorem in the case where $n$ is a power of $b>1$.
		$$ T(n) = \left\{
		\begin{aligned}
			&\Theta(n^{\log_ba})\quad &f(n)=\Theta(n^{\log_ba})\\
			&\Theta(n^{\log_ba})&f(n)=O(n^{\log_ba-\epsilon})\\
			&\Theta(f(n))\quad&f(n)\geq \frac{a}{c}f(n/b)
		\end{aligned}\right.
		$$
	\end{enumerate}

\section*{Ex.3 Ramanujam numbers}
	\begin{algorithm}
		\caption{Ramanujam numbers}
		\begin{algorithmic}[1]
			\Require Integer $n$
			\Ensure All the Ramanujam numbers smaller or equal to $n$.
			\State Initalize an array $Buckets$ with $n$ zeros.
			\State $Ramanujam\leftarrow\emptyset$
			\State $i\leftarrow0$
			\For{all $i < n^{1/3}$}
				\For{all $j < i$}
					\State $sum = i^3 + j^3$
					\If{$sum < n$}
						\State $Buckets[sum]+=1$
					\EndIf
				\EndFor
			\EndFor
			\For{each in $Buckets$}
				\State Append the index of the element with value of 2 in $Ramanujam$
			\EndFor
			\State \Return $Ramanujam$
		\end{algorithmic}
	\end{algorithm}
	$$1+2+3+4+\cdots+n^{1/3}=O(n^{2/3})$$
	$$O(n^{2/3})+O(n)=O(n)$$

\section*{Ex. 4 Critical thinking}
	\textbf{Two pirates left:} No.5 and No.6 are sharing the gold. No.5 will keep all 300 pieces and will be support by himself.\par
	\textbf{Three pirates left:} No.5 must be aganist to No.4 to get all of the gold. No.4 needs to give No.6 1 piece of gold so that No.6 will vote for him because if No.4 died, No.6 would get nothing. No.4 keeps 299 pieces.\par
	\textbf{Four pirates left:} No.4 must be aganist to No.3 to 299 pieces of gold. No.3 needs to give No.5 1 piece of gold so that No.5 will vote for him because if No.3 died, No.5 would get nothing. Then he gets enough support. And then he keeps 299 pieces.\par
	\textbf{Five pirates left:} No.3 must be aganist to No.2 to get 299 pieces of gold. No.2 needs to give No.4 and No.6 each one a piece of gold so that they will vote for him because of if No.2 died, they would get nothing. And then he keeps 298 pieces.\par
	\textbf{Six pirates left:} Similarly, No.1 298, No.3 1, No.5 1.

\end{document}
